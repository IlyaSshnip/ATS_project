\documentclass{beamer}

\usetheme{Warsaw}
\setbeamertemplate{caption}[numbered]
\setbeamertemplate{theorems}[numbered]

\begin{document}

\section{Multi-Period Kyle Model}
\subsection{Model Structure}

\begin{frame}
    \begin{itemize}
        \item One informed trader (a monopolist)
        \item There are $N$ auctions happening at times $t_k$, $0 = t_0 < t_1 < \cdots < t_N = 1$.
        \item For implementation purposes, we assume equally spaced intervals: $\Delta t_k = t_k - t_{k-1} = \frac{1}{N}$.
        \item The liquidation value of the asset $v \sim \mathcal{N}(p_0, \Sigma_0)$.
    \end{itemize}
\end{frame} 

\begin{frame}
    \begin{itemize}
        \item Quantity traded by noise traders is a \textit{Brownian motion process} $u_n = u(t_n)$.
        \item $\Delta u_n = u_n - u_{n-1}$ is normally distributed with zero mean and variance $\sigma^2_t\Delta t_n$.
        \item Quantities traded at one auction are independent of the quantities traded at other auctions.
    \end{itemize}
\end{frame}

\begin{frame}
    \begin{itemize}
        \item $x_n$ is the aggregate position of the insider after $n$-th auction, so that $\Delta x_n = x_n - x_{n-1}$ is the quantity traded at $n$-th auction.
        \item $p_n$ is the market clearing price at the $n$-th auction.
        \item When deciding what quantity to trade, the insider uses the liquidation value $v$ and past prices:
            \begin{equation}
                x_n = X_n(p_1, \ldots, p_{n-1}, v)
            \end{equation}
        \item Then, market makers set a market clearing price using current and all previous order flows:
            \begin{equation}
                p_n = P_n(x_1 + u_1, \ldots, x_n + u_n)
            \end{equation}
        \item Profit of the insider on the positions acquired at auctions $n, \ldots, N$:
            \begin{equation}
                \pi_n = \sum_{k=n}^{N}(v - p_k)x_k
            \end{equation}
    \end{itemize}
\end{frame}

\begin{frame}
    A \textit{sequential auction equilibrium} is defined such that:
    \begin{itemize}
        \item \textit{profit maximization} is achieved, i.e.~we find such trading rules $X=(X_1, \ldots, X_N)$ that the expected profit is maximized:
            \begin{equation}
                \mathbb{E}\{\pi_n(X, P) \mid p_1, \ldots, p_{n-1}, v\}
            \end{equation}
        \item \textit{market efficiency} condition holds:
            \begin{equation}
                p_n = \mathbb{E}\{v \mid x_1 + 
                u_1, \ldots, x_n + u_n\}
            \end{equation}
    \end{itemize}
\end{frame}

\begin{frame}
    Kyle considers a \textit{recursive linear equilibrium}, where $X_k$ and $P_k$ are linear and:
    \begin{align}
        \Delta x_n &= \beta_n(v - p_{n-1}) \Delta t_n \\
        \Delta p_n &= \lambda_n(\Delta x_n + \Delta u_n)
    \end{align}

    In this equilibrium, price increments are normally and independently distributed. Thus, the distribution function for the pricing process is characterized by a sequence of variance parameters measuring the volatility of price fluctuations:

        \begin{equation}
            \Sigma_n = Var \{v \mid x_1 + u_1, \ldots, x_n + u_n \}
        \end{equation}

    % Note that $\Sigma_0$ is just the variance of the initial prior price $p_0$.
\end{frame}

\subsection{Model Solution}

\begin{frame}
    The equilibrium defined above \textit{exists} and is \textit{unique}. Given $\Sigma_0$, the values of $\lambda_n$, $\beta_n$ and $\Sigma_n$ are a unique solution to the difference equation system, subject to $\alpha_N = 0$ and second order condition $\lambda_n (1 - \alpha_n\lambda_n) > 0$:

    \begin{align}
        \beta_n &= \dfrac{1 - 2\alpha_n\lambda_n}{2\lambda_n\Delta t_n (1 - \alpha_n\lambda_n)} \\
        \lambda_n &= \dfrac{\beta_n\Sigma_n}{\sigma_u^2} \\
        \alpha_{n-1} &= \dfrac{1}{4\lambda_{n}(1 - \alpha_{n}\lambda_{n})} \\
        \Sigma_{n-1} &= \dfrac{\Sigma_{n}}{1 - \beta_{n}\lambda_{n}\Delta t_{n}}
    \end{align}
\end{frame}

\begin{frame}
The system is solved backwards: first we provide a ``guess'' for $\Sigma_N$, and then for each auction in the order $N, N - 1, \ldots, 1$, Kyle showed that $\lambda_n$ is the middle root of the cubic equation:

\begin{equation}\label{eq:kyle_cubic}
    \left(1 - \dfrac{\lambda_n^2\sigma_u^2\Delta t_n}{\Sigma_n}\right)(1 - \alpha_n\lambda_n) = \dfrac{1}{2}
\end{equation}

So, for each iteration $n$, we already have $\alpha_n$ and $\Sigma_n$ calculated. We solve (\ref{eq:kyle_cubic}) for $\lambda_n$, and then calculate $\beta_n$, $\alpha_{n-1}$ and $\Sigma_{n-1}$ from $\lambda_n$. 

At the end, we compare our computed $\Sigma_0$ with the true value. If they are close, we solved the system, otherwise our guess for $\Sigma_N$ was incorrect. 

To find a correct $\Sigma_N$, we use methods like bisection or Newton-Raphson and minimize the difference between true and computed $\Sigma_0$.

\end{frame}

\subsection{Model Analysis}

\begin{frame}
    \begin{figure}\label{fig:lambda_on_N}
        \includegraphics[width=\linewidth]{figures/lambda_by_N.pdf}
        \caption{$\lambda_n$ depending on number of auctions $N$.}
    \end{figure}

    Notice how terminal value $\lambda_N$ increases with $N$. When we have a monopolistic informed trader, market depth decreases with more auctions held.
\end{frame}

\begin{frame}
    \begin{figure}\label{fig:beta_on_N}
        \includegraphics[width=\linewidth]{figures/beta_by_N.pdf}
        \caption{$\beta_n$ depending on number of auctions $N$.}
    \end{figure}

    Insider trading intensity increases with $N$, but slowly.

\end{frame}

\begin{frame}
    \begin{figure}\label{fig:sigma_on_N}
        \includegraphics[width=\linewidth]{figures/sigma_by_N.pdf}
        \caption{$\Sigma_n$ depending on number of auctions $N$.}

    \end{figure}

    Variance of prices, or the measure of how much information is not yet incorporated into prices, is decreasing slowly.
\end{frame}

\begin{frame}
    \begin{figure}\label{fig:lambda_on_uninf}
        \includegraphics[width=\linewidth]{figures/lambda_by_uninf.pdf}
        \caption{$\lambda_n$ vs $\sigma_u$.}
    \end{figure}

    Increasing the amount of noise trading increases market depth.
\end{frame}

\begin{frame}
    \begin{figure}\label{fig:beta_on_uninf}
        \includegraphics[width=\linewidth]{figures/beta_by_uninf.pdf}
        \caption{$\beta_n$ vs $\sigma_u$.}
    \end{figure}

    Insider trading intensity increases with noise trading increasing.
\end{frame}

\begin{frame}
    \begin{figure}\label{fig:sigma_on_uninf}
        \includegraphics[width=\linewidth]{figures/sigma_by_uninf.pdf}
        \caption{$\Sigma_n$ vs $\sigma_u$.}
    \end{figure}

    $\Sigma_n$, informativeness of trades, does not depend on noise trading volatility.
\end{frame}

\section{$M$ Informed Traders}

\subsection{Model Definition}

\begin{frame}
    \begin{itemize}
        \item \textbf{Holden} and \textbf{Subrahmanyam} proposed an extension of multi-period Kyle model by introducing $M$ informed traders.
        \item The model setup remains the same, except for $\Delta x_n$, which now means total order of \textit{all} informed traders at $n$-th auction.
        \item A unique linear equilibrium also exists for such a model.
    \end{itemize}
\end{frame}

\begin{frame}
    The equilibrium is defined by equations:

    \begin{align}
        \Delta x_n &= M\beta_n(v - p_{n-1}) \Delta t_n \\
        \Delta p_n &= \lambda_n(\Delta x_n + \Delta u_n) \\
        \Sigma_n &= Var \{v \mid x_1 + u_1, \ldots, x_n + u_n \}
    \end{align}
\end{frame}

\begin{frame}
    Difference equation system, with boundary conditions $\alpha_N = 0$ and second order condition $\lambda_n (1 - \alpha_n\lambda_n) > 0$:

    \begin{align}
        \beta_n &= \dfrac{1 - 2\alpha_n\lambda_n}{\lambda_n\Delta t_n (M(1 - 2\alpha_n\lambda_n) + 1)} \\
        \lambda_n &= \dfrac{M\beta_n\Sigma_n}{\sigma_u^2} \\
        \alpha_{n-1} &= \dfrac{1 - \alpha_n\lambda_n}{\lambda_{n}(M(1 - 2\alpha_{n}\lambda_{n}) + 1)^2} \\
        \Sigma_{n-1} &= \dfrac{\Sigma_{n}}{1 - M\beta_{n}\lambda_{n}\Delta t_{n}}
    \end{align}

    When $M=1$, we get the Kyle multi-period model.
\end{frame}

\subsection{Model Solution}

\begin{frame}
    \begin{itemize}
        \item This model cannot be solved with ``guessing'' approach, but Holden and Subrahmanyam found a better way, which solves model for $M$, including $M=1$.
        \item The method is explicit, and does not involve ``guessing'' $\Sigma_N$.
    \end{itemize}
\end{frame}

\begin{frame}
    Define $q_n = \alpha_n\lambda_n$. Starting from $q_N = 0$, solve in a backward manner, for every $n$, the cubic equation:

    \begin{equation}
        2M\dfrac{\Delta t_{n-1}}{\Delta t_n} q_{n-1}^3 - (M+1)\dfrac{\Delta t_{n-1}}{\Delta t_n}q_{n-1}^2 - 2k_n q_{n-1} + k_n= 0
    \end{equation}
    where
    \begin{equation}
        k_n = \dfrac{(1-q_n)^2}{(1 - 2q_n)(M(1-2q_n) + 1)^2}
    \end{equation}
    and choose the unique root that lies in the interval $(0, \frac{1}{2})$.
\end{frame}

\begin{frame}
    After that, starting from $\Sigma_0$, iterate forward and calculate parameters:

    \begin{align}
        \Sigma_n &= \dfrac{1}{M(1 - 2q_n) + 1}\Sigma_{n-1} \\
        \lambda_n &= \left(\dfrac{M\Sigma_n(1-2q_n)}{\Delta_t \sigma_u^2 (M(1-2q_n) + 1)}\right)^{\frac{1}{2}} \\
        \beta_n &= \left(\dfrac{(1-2q_n)\sigma_u^2}{\Sigma_n\Delta t_n (M(1-2q_n) + 1)}\right)^{\frac{1}{2}}
    \end{align}

\end{frame}

\subsection{Model Analysis}

\begin{frame}
    \begin{figure}\label{fig:lambda_by_MN}
        \includegraphics[width=\linewidth]{figures/lmbd_by_M_N.pdf}
        \caption{$\lambda_N$ vs $N$ for different $M$.}
    \end{figure}

    As soon as we add competition, market depth increases a lot with each added informed trader.
\end{frame}

\begin{frame}
    \begin{figure}\label{fig:beta_by_MN}
        \includegraphics[width=\linewidth]{figures/beta_by_M_N.pdf}
        \caption{$\beta_N$ vs $N$ for different $M$.}
    \end{figure}

    Trading intensity explodes in comparison to $M=1$ case. Informed traders are much more aggressive.

\end{frame}

\begin{frame}
    \begin{figure}\label{fig:sigma_by_MN}
        \includegraphics[width=\linewidth]{figures/sigma_by_M_N.pdf}
        \caption{$\Sigma_N$ vs $N$ for different $M$.}
    \end{figure}

    Private information is incorporated very fast.
\end{frame}

\begin{frame}
    \begin{figure}\label{fig:lambda_by_M}
        \includegraphics[width=\linewidth]{figures/lmbd_by_M.pdf}
        \caption{$\lambda_n$ vs $M$, $N=3$.}
    \end{figure}

    Market is very liquid at the final auction.
\end{frame}

\begin{frame}
    \begin{figure}\label{fig:beta_by_M}
        \includegraphics[width=\linewidth]{figures/beta_by_M.pdf}
        \caption{$\beta_n$ vs $M$, $N=3$.}
    \end{figure}

    Final auction trading intensity is rapidly increasing with increasing $M$.
\end{frame}

\begin{frame}
    \begin{figure}\label{fig:sigma_by_M}
        \includegraphics[width=\linewidth]{figures/sigma_by_M.pdf}
        \caption{$\Sigma_n$ vs $M$, $N=3$.}
    \end{figure}

    Already when we have 4 competitors, there is no hidden information left to trade on.
\end{frame}

\end{document}
