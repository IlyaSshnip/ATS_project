\documentclass{beamer}

\usetheme{Warsaw}

\begin{document}

\section{Multi-Period Kyle Model}
\subsection{Model Structure}

\begin{frame}
    \begin{itemize}
        \item As before, we have one informed trader
        \item There are $N$ auctions happening at times $t_k$, $0 = t_0 < t_1 < \cdots < t_N = 1$.
        \item For implementation purposes, we assume equally spaced intervals: $\Delta t_k = t_k - t_{k-1} = \frac{1}{N}$.
        \item The liquidation value of the asset $\tilde{v} \sim \mathcal{N}(p_0, \Sigma_0)$.
    \end{itemize}
\end{frame} 

\begin{frame}
    \begin{itemize}
        \item Quantity traded by noise traders is a \textit{Brownian motion process} $\tilde{u}_n = \tilde{u}(t_n)$.
        \item $\Delta \tilde{u}_n = \tilde{u}_n - \tilde{u}_{n-1}$ is normally distributed with zero mean and variance $\sigma^2_t\Delta t_n$.
        \item Quantities traded at one auction are independent of the quantities traded at other auctions.
    \end{itemize}
\end{frame}

\begin{frame}
    \begin{itemize}
        \item $\tilde{x}_n$ is the aggregate position of the insider after $n$-th auction, so that $\Delta \tilde{x}_n = \tilde{x}_n - \tilde{x}_{n-1}$ is the quantity traded at $n$-th auction.
        \item $\tilde{p}_n$ is the market clearing price at the $n$-th auction.
        \item When deciding what quantity to trade, the insider uses the liquidation value $\tilde{v}$ and past prices:
            \begin{equation}
                \tilde{x}_n = X_n(\tilde{p}_1, \ldots, \tilde{p}_{n-1}, \tilde{v})
            \end{equation}
        \item Then, market makers set a market clearing price using current and all previous order flows:
            \begin{equation}
                \tilde{p}_n = P_n(\tilde{x}_1 + \tilde{u_1}, \ldots, \tilde{x}_n + \tilde{u}_n)
            \end{equation}
        \item Profit of the insider on the positions acquired at auctions $n, \ldots, N$:
            \begin{equation}
                \tilde{\pi}_n = \sum_{k=n}^{N}(\tilde{v} - \tilde{p_k})\tilde{x}_k
            \end{equation}
    \end{itemize}
\end{frame}

\begin{frame}
    A \textit{sequential auction equilibrium} is defined such that:
    \begin{itemize}
        \item \textit{profit maximization} is achieved, i.e.~we find such trading rules $X=(X_1, \ldots, X_N)$ that the expected profit is maximized:
            \begin{equation}
                \mathbb{E}\{\tilde{\pi}_n(X, P) \mid \tilde{p}_1, \ldots, \tilde{p}_{n-1}, \tilde{v}\}
            \end{equation}
        \item \textit{market efficiency} holds:
            \begin{equation}
                \tilde{p}_n = \mathbb{E}\{\tilde{v} \mid \tilde{x}_1 + 
                \tilde{u}_1, \ldots, \tilde{x}_n + \tilde{u}_n\}
            \end{equation}
    \end{itemize}
\end{frame}

\begin{frame}
    Kyle considers a \textit{recursive linear equilibrium}, where $X_k$ and $P_k$ are linear and:
    \begin{align}
        \Delta \tilde{x}_n &= \beta_n(\tilde{v} - \tilde{p}_{n-1}) \Delta t_n \\
        \Delta \tilde{p}_n &= \lambda_n(\Delta \tilde{x}_n + \Delta \tilde{u}_n)
    \end{align}

    In this equilibrium, price increments are normally and independently distributed. Thus, the distribution function for the pricing process is characterized by a sequence of variance parameters measuring the volatility of price fluctuations:

        \begin{equation}
            \Sigma_n = Var \{\tilde{v} \mid \tilde{x}_1 + \tilde{u_1}, \ldots, \tilde{x}_n + \tilde{u}_n \}
        \end{equation}

    % Note that $\Sigma_0$ is just the variance of the initial prior price $p_0$.
\end{frame}

\end{document}